\begin{challenge}
    \chatitle{Dynamic Linked Containers with \texttt{std::list} in C++}
    \begin{chadescription}
    In C++, the \texttt{std::list} class represents a doubly linked list, a dynamic container that allows efficient insertion and deletion of elements at any position. 
    Unlike arrays or \texttt{std::array}, \texttt{std::list} is not limited by a fixed size, and it can grow or shrink dynamically as elements are added or removed.

    The \texttt{std::list} container is especially useful when you need frequent insertions and deletions in the middle of the collection, as these operations are more efficient compared to other containers like \texttt{std::vector}. 
    With built-in functions like \texttt{push_back()}, \texttt{push_front()}, \texttt{insert()}, and \texttt{erase()}, it abstracts the complexities of pointer management inherent in manually implemented linked lists in C. 

    In this challenge, you will explore how to use \texttt{std::list} for dynamic data storage and manipulation. 
    You will also compare it to a manually implemented linked list in C to understand the advantages of using this STL container. 
    By the end, you will have a solid grasp of when and why to use \texttt{std::list} in C++ programs.
    \end{chadescription}

    \begin{task}
        Write a simple program using \texttt{std::list} that performs the following steps:
        \begin{enumerate}
            \item Declare an \texttt{std::list} to store integer values.
            \item Add the numbers \texttt{10}, \texttt{20}, and \texttt{30} to the list using \texttt{push_back()}.
            \item Print all elements of the list using a range-based \texttt{for} loop.
            \item Add the number \texttt{5} to the front of the list using \texttt{push_front()}.
            \item Print the updated list to the terminal.
        \end{enumerate}

        Save your program and run it to ensure the output matches your expectations.

        \begin{questions}
            \item What is the difference between \texttt{push_back()} and \texttt{push_front()} in \texttt{std::list}?
            \item How does the way you add elements to a list differ from adding elements to a raw C-style array?
            \item Why is it easier to insert elements at the front or back of an \texttt{std::list} compared to a \texttt{std::vector} or \texttt{std::array}?
        \end{questions}
    \end{task}

    \begin{task}
        Write a program using \texttt{std::list} that performs the following steps:
        \begin{enumerate}
            \item Declare and initialize an \texttt{std::list} with the values \texttt{1, 2, 3, 4, 5}.
            \item Implement a function named \texttt{printList} that takes an \texttt{std::list} as a parameter and prints its elements to the terminal.
            \item Insert the value \texttt{10} into the list at the third position.
            \item Remove the second element from the list.
            \item Call \texttt{printList} after each modification to observe the changes.
        \end{enumerate}

        Save your program, compile it, and ensure that it runs correctly.

        \begin{questions}
            \item What challenges would arise when inserting or deleting elements in a manually implemented linked list in C, and how does \texttt{std::list} address these?
            \item How does \texttt{std::list}'s iterator simplify accessing specific positions in the list compared to using indices in an array?
            \item What happens to the memory of an element in \texttt{std::list} after you remove it using \texttt{erase()}?
        \end{questions}
    \end{task}

    \begin{task}
        Write a program using \texttt{std::list} that performs the following steps:
        \begin{enumerate}
            \item Declare and initialize an \texttt{std::list} of integers with the values \texttt{3, 1, 4, 1, 5, 9, 2}.
            \item Implement a function named \texttt{findMin} that takes an \texttt{std::list} as a parameter and returns the smallest element.
            \item Reverse the order of elements in the list manually by swapping elements from the front and back using iterators.
            \item Print the list after reversing it and display the smallest element using the \texttt{findMin} function.
        \end{enumerate}

        Save your program, compile it, and ensure it runs correctly. This task requires careful use of iterators to manipulate the list dynamically.

        \begin{questions}
            \item How does using iterators to reverse a list compare to using pointers in a manually implemented linked list in C?
            \item What challenges might arise when working with both forward and reverse iterators, and how did you address these?
            \item Why is \texttt{std::list} particularly well-suited for dynamic operations like reversing compared to \texttt{std::array} or \texttt{std::vector}?
        \end{questions}
    \end{task}

    \begin{advise}
        Working with \texttt{std::list} helps you understand how dynamic data containers can simplify memory management and complex data manipulations in C++. 
        Unlike raw C-style linked lists, \texttt{std::list} provides built-in methods that make adding, removing, and rearranging elements much safer and more efficient. 
        Through this challenge, you have explored the use of iterators, dynamic resizing, and the flexibility of \texttt{std::list} in handling ordered collections. 
        Remember to choose \texttt{std::list} when you need frequent insertions and deletions but don't require random access, as other containers may be more efficient for those use cases. 
        With practice, you'll find that \texttt{std::list} is a versatile tool for solving many programming challenges.
    \end{advise}
\end{challenge}
